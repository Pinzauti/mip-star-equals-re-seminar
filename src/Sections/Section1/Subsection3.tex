\subsection{Schmidt decomposition}

\subsection{Entanglement}
Consider two arbitrary quantum systems $Q_1$ and $Q_2$, with respective Hilbert spaces $\mathcal{H}_{Q_1}$ and $\mathcal{H}_{Q_2}.$ The Hilbert space of the composite system is the tensor product: 
\begin{equation*}
\mathcal{H}_{Q_1} \otimes \mathcal{H}_{Q_2}\,,
\end{equation*}
if the first system is in state $\ket{\psi}_{Q_1}$ and the second in state $\ket{\psi}_{Q_2}$ the state of the composite system is
\begin{equation}
    \ket{\psi}_{Q_1} \otimes \ket{\psi}_{Q_2}\,.
\end{equation}
States of the composite system that can be represented in this form are called \emph{separable states} while
\begin{defn}
A composite system such that $\ket{QR} \neq \otimes_i \ket{Q_i}$ is an \emph{entangled state}~\cite{verrucchi}.
\end{defn}

\subsubsection{EPR pairs}
We define the EPR pairs, specific quantum states of two qubits that represent the simplest (and maximal) examples of quantum entanglement. Those states are a basis of a $\dim{2}$ Hilbert space.

Given the states $\ket{x} \otimes \ket{y} = \ket{0} \otimes \ket{0}, \ket{0} \otimes \ket{1}, \ket{1} \otimes \ket{0}, \ket{1} \otimes \ket{1}$ we define the EPR states as:
\begin{equation}
    \ket{\beta_{x y}} \equiv \frac{\ket{0} \otimes \ket{y} +(-1)^{x} \ket{1} \otimes \ket{\bar{y}}}{\sqrt{2}}.
\end{equation}
\subsection{Bell's theorem}
