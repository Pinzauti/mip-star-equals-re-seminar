
\subsection{Von Neumann algebras}
We build on the concepts of Banach space and (abstract) $C^{*}$-algebra. 

\begin{defn}
A Banach space is a complex Banach space and a morphism of Banach spaces is a short linear map (a complex-linear map of norm at most 1); a $C^{*}$-algebra is a complex unital $C^{*}$-algebra, and a morphism of $C^{*}$-algebras is a unital $*$-homomorphism (which is necessarily also a short linear map).
Given a Banach space $A$, a predual of $A$ is a Banach space $V$ whose dual Banach space $V^{*}$ is isomorphic to $A$ :
$$
V^{*} \stackrel{i}{\rightarrow} A
$$
. Similarly, given a morphism $f: A \rightarrow B$ (properly, with $A$ and $B$ so equipped), a predual of $f$ is a morphism $t: W \rightarrow V$ whose dual is isomorphic to $f$ :
$$
\begin{array}{ccc}
V^{*} & \stackrel{i}{\rightarrow} & A \\
t^{*} \downarrow & & \downarrow f \\
W^{*} & \stackrel{j}\rightarrow & B
\end{array}
$$
~\cite{vonNeuma44:online}
\end{defn}
This means that a $W^{*}$-algebra is a $C^{*}$-algebra that is equipped with a predual, and a $W^{*}$-homomorphism is a $C^{*}$-homomorphism that admits a predual. In this way, the category of $W^{*}$-algebras becomes a subcategory of the category of $C^{*}$-algebras.


\subsection{The problem}

Von Neumann algebras can be seen as rings of operators. A fundamental goal in operator algebras is to provide a classification of von Neumann algebras. We could prove that classifying von Neumann algebras is equal to understanding their factors.

We know that factors of von Neumann algebras belongs to one of the following species: type I, type II, and type III. 

The problem state as follows:

\begin{problem}
    Does every type $II_1$ von Neumann factor embeds into an ultrapower of the hyperfinite $II_1$ factor?~\cite{mipre}
\end{problem}

Thanks to the work of~\cite{Tsirelson_1} and~\cite{Tsirelson_2} we have a mapping of Connes' embedding problem to Tsirelson's problem:

\begin{theorem}
If and only if $C_{q a}(n, k) = C_{q c}(n, k)$ then every type $II_1$ von Neumann factor embeds into an ultrapower of the hyperfinite $II_1$ factor.~\cite{mipre}
\end{theorem}

Because we now know that Tsirelson's problem has a negative answer, we can give a negative answer also to the Connes' embedding problem.

