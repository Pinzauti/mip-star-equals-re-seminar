
\subsection{Von Neumann algebras}
We build on the concepts of Banach space and (abstract) $C^{*}$-algebra. In this definition, a Banach space is a complex Banach space and a morphism of Banach spaces is a short linear map (a complex-linear map of norm at most 1); a $C^{*}$-algebra is a complex unital $C^{*}$-algebra, and a morphism of $C^{*}$-algebras is a unital $*$-homomorphism (which is necessarily also a short linear map). Note in particular that an isomorphism of either must be an isometry.
Given a Banach space $A$, a predual of $A$ is a Banach space $V$ whose dual Banach space $V^{*}$ is isomorphic to $A$ :
$$
V^{*} \stackrel{i}{\rightarrow} A
$$
(or more properly, equipped with such an isomorphism $i$ ). Similarly, given a morphism $f: A \rightarrow B$ (properly, with $A$ and $B$ so equipped), a predual of $f$ is a morphism $t: W \rightarrow V$ whose dual is isomorphic to $f$ :
$$
\begin{array}{ccc}
V^{*} & \stackrel{i}{\rightarrow} & A \\
t^{*} \downarrow & & \downarrow f \\
W^{*} & \stackrel{j}\rightarrow & B
\end{array}
$$
With these preliminaries, a $W^{*}$-algebra or ("abstract") von Neumann algebra is a $C^{*}$-algebra that admits a predual (or more properly, equipped with one), and a $W^{*}$-homomorphism is a $C^{*}$-homomorphism that admits a predual. In this way, the category of $W^{*}$-algebras becomes a subcategory of the category of $C^{*}$-algebras.


\subsection{The problem}

Von Neumann algebras can be seen as rings of operators. One of the most important goals in operator algebras has been to provide a classification of von Neumann algebras and we can prove that classifying von Neumann algebras reduces to understanding their factors, the atoms out of which all von Neumann algebras are built.

It is known that factors of von Neumann algebras come in one of three species: type I, type II, and type III. 

The problem state as follows:

\begin{problem}
    Does every type $II_1$ von Neumann factor embeds into an ultrapower of the hyperfinite $II_1$ factor?
\end{problem}

Following the work of~\cite{Tsirelson_1} and~\cite{Tsirelson_2} we can provide a mapping of Connes' embedding conjecture from the theory of von Neumann algebras to Tsirelson's problem:

\begin{theorem}
If and only if $C_{q a}(n, k) = C_{q c}(n, k)$ then every type $II_1$ von Neumann factor embeds into an ultrapower of the hyperfinite $II_1$ factor.
\end{theorem}

Because we now know that Tsirelson's problem has a negative answer, we can give a negative answer also to the Connes' embedding problem.

