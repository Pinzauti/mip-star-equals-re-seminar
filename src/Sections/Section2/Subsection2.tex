\subsection{The class MIP}

\begin{defn}
    Let the provers $P_{1}, \ldots, P_{i}$ be computationally unbounded Turing machines. Let the verifier $V$ be a PPT machine. We allow each $P_{i}$ to communicate with $V$ and vice versa during the course of the protocol, bu
    1
    we do not allow the $P_{i}$ to communicate with one another. Then we say $L \in$ MIP if $x \in L \Longrightarrow \operatorname{Pr}\left[P_{1}, \ldots, P_{k}\right.$ convince $\mathrm{V}$ to accept $\left.\mathrm{x}\right]=1$
    $x \notin L \Longrightarrow \forall P_{1}^{\prime}, \ldots, P_{k}^{\prime}, \operatorname{Pr}\left[P_{1}^{\prime}, \ldots, P_{k}^{\prime}\right.$ convince $\mathrm{V}$ to accept $\left.\mathrm{x}\right] \leq 1 / 2$
    where the $P^{\prime}$ notation denotes a dishonest prover and where the constants are arbitrary.
\end{defn}

\begin{theorem}
MIP=NEXPTIME~\cite{topicsin}.
\end{theorem}

This shows how MIP is drastically more powerful of IP, allowing us to verify polynomially all the problems in NEXPTIME


\subsection{The class MIP*}


\subsection{Nonlocal games}

\begin{theorem}
MIP=MIP(2,1).
\end{theorem}
\subsection{The class RE}


Let us make a little bit of order and recollect all the known result from~\cite{papadimitriou1994computational} plus what we got since here:

\begin{equation}
\begin{split}
    &L \subseteq \text{NL} \subseteq P \subseteq \text{NP} \subseteq \text{PH} \subseteq \text{PSPACE} = \text{NPSPACE} = \text{IP} \subseteq \text{EXPTIME} \subseteq  \\
    &\subseteq \text{NEXPTIME} = \text{MIP} \subseteq \text{EXPSPACE} = \text{NEXPSPACE} \subseteq R \subseteq \text{RE}.
\end{split}
\end{equation}

Given the title of this seminar, it should't been difficult to guess the place of MIP* in the hierarchy.

\subsection{CHSH game}
