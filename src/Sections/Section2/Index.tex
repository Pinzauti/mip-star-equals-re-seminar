\section{Tsirelson's problem}
\epigraph{Natura non facit saltus.}{Gottfried Wilhelm Leibniz}

Before stating the problem, we need to state a set of definitions. Despite the sometimes heavy notation, what we are going to define is pretty simple. We have two separated\footnote{Very separated: through all the seminar we are going to assume both the measurements to occur, using the language of relativity, in a in a causally disconnected manner.} systems $A$ and $B$ that receive two inputs $x,y$ from a given set $n$ and produce\footnote{The output simply depends on the measure performed by those systems on the inputs.} two (separated) outputs $a,b \in k$. We want to describe the behavior of the system, namely its $(n,k)-correlation$, which is a probability $p(a,b | x,y)$. As we don't want to be tied to the specific inputs, we focus in set of correlations, namely $p(a,b | x,y) \forall x, y$. 

It goes without saying that the depending on the formalism and the models that we use, this correlation changes. In the following sections we are going to describe the different type of correlations and most importantly the relationship between them.

\subsection{Classical correlation set}
\subsection{Quantum correlation set}

\begin{defn}
    For integer $n, k \geq 2$ define the quantum (spatial) correlation set $C_{q s}(n, k)$ as the subset of $\mathbb{R}^{n^{2} k^{2}}$ such that $\left(p_{a b x y}\right) \in C_{q s}(n, k)$ if and only if there exist separable Hilbert spaces $\mathcal{H}_{\mathrm{A}}$ and $\mathcal{H}_{\mathrm{B}}$, for every $x \in\{1, \ldots, n\}$ (resp. $y \in\{1, \ldots, n\}$ ), a collection of projections $\left\{A_{a}^{x}\right\}_{a \in\{1, \ldots, k\}}$ on $\mathcal{H}_{\mathrm{A}}$ (resp. $\left\{B_{b}^{y}\right\}_{b \in\{1, \ldots, k\}}$ on $\mathcal{H}_{\mathrm{B}}$ ) that sum to identity, and a state (unit vector) $\psi \in \mathcal{H}_{\mathrm{A}} \otimes \mathcal{H}_{\mathrm{B}}$ such that
    \begin{equation}
    \forall x, y \in\{1,2, \ldots, n\}, \quad \forall a, b \in\{1,2, \ldots, k\}, \quad p_{a b x y}=\bra{\psi}\left(A_{a}^{x} \otimes B_{b}^{y}\right) \ket{\psi} .
    \end{equation}.
\end{defn}

Note that due to the normalization conditions on $\psi$ and on $\left\{A_{a}^{x}\right\}$ and $\left\{B_{b}^{y}\right\}$, for each $x, y,\left(p_{a b x y}\right)$ is a probability distribution on $\{1,2, \ldots, k\}^{2}$. By taking direct sums it is easy to see that the set $C_{q s}(n, k)$ is convex. Let $C_{q a}(n, k)$ denote its closure (it is known that $C_{q s}(n, k) \neq C_{q a}(n, k)$, see [Slo19a]).~\cite{mipre}.

\subsection{Quantum commuting correlation set}
\subsection{The problem}

\subsection{Schmidt decomposition}

\subsection{Entanglement}
Consider two arbitrary quantum systems $Q_1$ and $Q_2$, with respective Hilbert spaces $\mathcal{H}_{Q_1}$ and $\mathcal{H}_{Q_2}.$ The Hilbert space of the composite system is the tensor product: 
\begin{equation*}
\mathcal{H}_{Q_1} \otimes \mathcal{H}_{Q_2}\,,
\end{equation*}
if the first system is in state $\ket{\psi}_{Q_1}$ and the second in state $\ket{\psi}_{Q_2}$ the state of the composite system is
\begin{equation}
    \ket{\psi}_{Q_1} \otimes \ket{\psi}_{Q_2}\,.
\end{equation}
States of the composite system that can be represented in this form are called \emph{separable states} while
\begin{defn}
A composite system such that $\ket{QR} \neq \otimes_i \ket{Q_i}$ is an \emph{entangled state}~\cite{verrucchi}.
\end{defn}

\subsubsection{EPR pairs}

\begin{equation}
    \left|\beta_{x y}\right\rangle \equiv \frac{|0, y\rangle+(-1)^{x}|1, \bar{y}\rangle}{\sqrt{2}}
\end{equation}
\subsection{Bell's theorem}
