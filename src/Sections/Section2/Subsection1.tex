
Before introducing the class IP, we quickly remind (one of) the definition of the class NP.

\begin{defn}
d
\end{defn}
\subsection{The classes IP}
The class IP is an extension of the class NP. Two key differences are introduced: first, we allow randomized verification procedure by relaxing the completeness and soundness condition. Then we allow interactive verification, i.e. we introduce a prover and a verifier.
\begin{defn}
    Consider the following model:
    \begin{itemize}
        \item a deterministic unbounded time prover $P$,
        \item a randomized polynomial time verifier $V$,
        \item a pair of conversation tapes on which $P$ and $V$ exchange information.
    \end{itemize}
    An interactive proof system (IPS) for a language $L$, is a protocol between $P$ and $V$ where
    \begin{itemize}
        \item $P$ and $V$ are given an input $x$,
        \item through an exchange of messages, $P$ tries to prove to $V$ that $x \in L$,
        \item at the end of the interaction, $V$ outputs either "accept" if the proof is satisfactory or "reject" if not.
    \end{itemize}
    We require that
    \begin{description}
\item[completeness:] if both $P$ and $V$ follow the protocol and $x \in L$, then
$$
\operatorname{Pr}[V \text { accepts } x] \geq c,
$$
\item[soundness:] if $x \notin L$ and $V$ follows the protocol, then, regardless of what $P$ does,
$$
\operatorname{Pr}[V \text { accepts } x] \leq s.
$$
    \end{description}
    where $c$ is the completeness parameter and $s$ is the soundness parameter. A common setting is to take $c=\frac{2}{3}$ and $s=\frac{1}{3}$~\cite{randomness}.
\end{defn}

\begin{defn}
    The class \textbf{IP} is the class of languages L such that there exists an IPS for L~\cite{randomness}.
\end{defn}
In an interactive proof, an all-powerful prover is trying to convince a skeptical, but computationally limited, verifier that a string $z$ (known to both) lies in the set $L$, even when it may be that in fact $z \notin L$. By interactively interrogating the prover, the verifier can reject false claims, i.e. determine with high statistical confidence whether $z \in L$ or not. Importantly, the verifier is allowed to probabilistically and adaptively choose its messages to the prover.


An example of an IP-complete problem is True quantified Boolean formula.

\begin{theorem}
 IP = PSPACE~\cite{ipequalspspace}.
\end{theorem}

This means that we can polynomially verify and problem in PSPACE, using an interactive proof system.


