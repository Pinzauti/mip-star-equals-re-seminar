
\subsection{Halting problem}

\begin{theorem}
    The halting problem $H$ is RE-complete~\cite{Algorithms}.
\begin{proof}
    Let $L$ be any recursively enumerable language. Assume $M$ accepts $L$. Clearly, one can decide whether $x \in L$ by asking if $M: x \in H$. This reduction is clearly computable. Hence all recursively enumerable languages are reducible to $H$.
\end{proof}
\end{theorem}

What we are going to do then is 
\subsection{Sketch of the proof}
\subsubsection{Recursive compression of non-local games}
\subsubsection{The compression theorem}

\subsection{Consequences on Tsirelson's problem}
As proved in~\ref{subsection:quantum-games} we have an isomorphism between the optimization problem $\operatorname{val^{*}}(\mathfrak{G})$ and the class MIP*(2, 1) and therefore the whole MIP*. But we just (kind of) proved that MIP*=RE.
What does that means?

Again, we are not going to be concerned with a specific proof, but we will just sketch it.


