\subsection{The class MIP}
In \emph{Multiprover Interactive Proofs} a polynomial-time verifier can interact with two (or more) provers to decide
whether a given instance $x$ is in a language $L$ or not. In this setting, after the verifier and all the provers receive the common input z, the provers are not allowed to communicate with each other, and the verifier “cross-interrogates” the provers in order to decide if $z \subseteq L$. The provers may coordinate a joint strategy ahead of time, but once the protocol begins the provers can only interact with the verifier. Formally:
\begin{defn}
    Let the provers $P_{1}, \ldots, P_{i}$ be computationally unbounded Turing machines. Let the verifier $V$ be a PPT\footnote{Probabilistic polynomial-time machines, i.e. Turing machines that are polynomially-bound and probabilistic.} machine. We allow each $P_{i}$ to communicate with $V$ and vice versa during the course of the protocol, but we do not allow the $P_{i}$ to communicate with one another. 
    
    We say $L \in$ MIP if 
    \begin{description}
\item[Completness: ]$x \in L \Rightarrow \operatorname{Pr}\left[P_{1}, \ldots, P_{k}\right.$ convince $\mathrm{V}$ to accept $\left.\mathrm{x}\right]=1$
\item[Soundness: ]$x \notin L \Rightarrow \forall P_{1}^{\prime}, \ldots, P_{k}^{\prime} \quad \operatorname{Pr}\left[P_{1}^{\prime}, \ldots, P_{k}^{\prime}\right.$ convince $\mathrm{V}$ to accept $\left.\mathrm{x}\right] \leq 1 / 2$
    \end{description}   
    where the $P^{\prime}$ notation denotes a dishonest prover and where the constants are arbitrary.
\end{defn}

An example of a MIP problem is the \textsc{gap-maxcut} problem.
\begin{theorem}
MIP=NEXPTIME~\cite{topicsin}.
\end{theorem}

This shows how MIP is drastically more powerful of IP, allowing us to verify polynomially all the problems in NEXPTIME.


\subsubsection{The class MIP*}
The definition of the class MIP* is exactly the same as MIP, except that the provers can share arbitrarily many entangled qubits. The verifier is classical, as are all messages between the provers and verifier. 

Note that this does not mean that the provers can communicate with each other. What does it means is that some of the qubits in their possession are entangled with the qubits of the other (for obvious reasons, the dimension of the entanglement is the same for both the parties). Does this mean that they can share information? No, it does not. As stated already many times, the provers are located far enough to make the measurements operations disconnected, and we can prove that quantum teletransportation is impossible without an exchange of classical information~\cite{NielsenChuang}. However the fact that the provers can not communicate does not mean that entanglement gives no power to MIP*, it definitely does, as we are going to see.
Note also that there is no bound on the dimension of the entanglement.

We have a fundamental result which already suggests the power of MIP*~\cite{mipre}:

\begin{theorem}
    \begin{equation}
\text{MIP} \neq \text{MIP}^{*}.
    \end{equation}
\end{theorem}

\subsection{Nonlocal games}

We will only be concerned with multiprover interactive proof systems
that consist of a single round of communication with two provers: the verifier first sends its questions to
each of the provers, the provers respond with their answers, and the verifier decides whether to accept or
reject. The class of problems that admit such interactive proofs is denoted MIP(2, 1).

This is because we can prove that~\cite{mipre}:
\begin{theorem}
    \begin{equation}
    \text{MIP}=\text{MIP}(2,1).
    \end{equation}
\end{theorem}

\subsubsection{Classical games}

\subsubsection{Quantum games}\label{subsection:quantum-games}

\begin{defn} A two-player one-round game $\mathfrak{G}$ is specified by a tuple $(\mathcal{X}, \mathcal{Y}, \mathcal{A}, \mathcal{B}, \mu, D)$ where
    \begin{itemize}
        \item $\mathcal{X}$ and $\mathcal{Y}$ are finite sets (called the question alphabets),
        \item $\mathcal{A}$ and $\mathcal{B}$ are finite sets (called the answer alphabets),
        \item $\mu$ is a probability distribution over $\mathcal{X} \times \mathcal{Y}$ (called the question distribution), and
        \item $D: \mathcal{X} \times \mathcal{Y} \times \mathcal{A} \times \mathcal{B} \rightarrow\{0,1\}$ is a function (called the decision predicate).
    \end{itemize}
\end{defn}

A language $L$ is in $\operatorname{MIP}^{*}(2,1)$ if and only if there is an efficient mapping from instances $z \in\{0,1\}^{*}$ to nonlocal games $\mathfrak{G}_{z}$ such that if $z \in L$, then $\operatorname{val}^{*}\left(\mathfrak{G}_{z}\right) \geq 2 / 3$ and otherwise $\operatorname{val}^{*}\left(\mathfrak{G}_{z}\right) \leq 1 / 3$. Here, for an $n$-question, $k$-answer game $\mathfrak{G}$, we let val ${ }^{*}(\mathfrak{G})$ denote its
We therefore have an isomorphism between the quantum game $\mathfrak{G}$ and the class MIP*(2, 1).

\begin{defn}
    \begin{equation}
    \operatorname{val}(\mathfrak{G})=\sup _{p \in C_{c}(n, k)} \sum_{x, y} \mu(x, y) \sum_{a, b} D(x, y, a, b) p_{a b x y},
    \end{equation}
\end{defn}

\subsubsection{The class RE}

\begin{defn}
A language $L$ is in RE, \emph{recursively enumerable} if exists a turing machine M which:
\begin{itemize}
\item If $x \in L$, then $M$ accept $x$.
\item if $x \notin L$, then $M$ reject $x$ or $\uparrow$.
\end{itemize}
\end{defn}

Let us make a little bit of order and recollect all the known result from~\cite{papadimitriou1994computational} plus what we got since here:

\begin{equation}
\begin{split}
    &L \subseteq \text{NL} \subseteq P \subseteq \text{NP} \subseteq \text{PH} \subseteq \text{PSPACE} = \text{NPSPACE} = \text{IP} \subseteq \text{EXPTIME} \subseteq  \\
    &\subseteq \text{NEXPTIME} = \text{MIP} \subseteq \text{EXPSPACE} = \text{NEXPSPACE} \subseteq R \subseteq \text{RE}.
\end{split}
\end{equation}

Given the title of this seminar, it should't been difficult to guess the place of MIP* in the hierarchy, and therefore how much powerful multiprover interactive proof with entangled are.

\subsection{CHSH game}
