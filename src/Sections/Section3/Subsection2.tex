\subsection{Quantum correlation set}
\subsection{Quantum commuting correlation set}
\subsection{The problem}
For integer $n, k \geq 2$ define the quantum (spatial) correlation set $C_{q s}(n, k)$ as the subset of $\mathbb{R}^{n^{2} k^{2}}$ that contains all tuples $\left(p_{a b x y}\right)$ representing families of bipartite distributions that can be locally generated in non-relativistic quantum mechanics. Formally, $\left(p_{a b x y}\right) \in C_{q s}(n, k)$ if and only if there exist separable Hilbert spaces $\mathcal{H}_{\mathrm{A}}$ and $\mathcal{H}_{\mathrm{B}}$, for every $x \in\{1, \ldots, n\}$ (resp. $y \in\{1, \ldots, n\}$ ), a collection of projections $\left\{A_{a}^{x}\right\}_{a \in\{1, \ldots, k\}}$ on $\mathcal{H}_{\mathrm{A}}$ (resp. $\left\{B_{b}^{y}\right\}_{b \in\{1, \ldots, k\}}$ on $\mathcal{H}_{\mathrm{B}}$ ) that sum to identity, and a state (unit vector) $\psi \in \mathcal{H}_{\mathrm{A}} \otimes \mathcal{H}_{\mathrm{B}}$ such that
\begin{equation}
\forall x, y \in\{1,2, \ldots, n\}, \quad \forall a, b \in\{1,2, \ldots, k\}, \quad p_{a b x y}=\bra{\psi}\left(A_{a}^{x} \otimes B_{b}^{y}\right) \ket{\psi} .
\end{equation}
Note that due to the normalization conditions on $\psi$ and on $\left\{A_{a}^{x}\right\}$ and $\left\{B_{b}^{y}\right\}$, for each $x, y,\left(p_{a b x y}\right)$ is a probability distribution on $\{1,2, \ldots, k\}^{2}$. By taking direct sums it is easy to see that the set $C_{q s}(n, k)$ is convex. Let $C_{q a}(n, k)$ denote its closure (it is known that $C_{q s}(n, k) \neq C_{q a}(n, k)$, see [Slo19a]).