
Before introducing the class IP, we quickly remind (one of) the definition of the class NP.

\begin{defn}
    A language $L$ admits efficiently verifiable proofs if there exists a polynomial-time Turing machine $V$ that satisfies the following two properties: 
    \begin{description}\label{descr:np}
        \item[Completeness: ]for any $z \in L$ there is a string $\pi$ such that $V(z, \pi)$ returns 1 (i.e. $V$ "accepts").
        \item[Soundness: ]for any $z \notin L$ there is no string $\pi$ such that $V(z, \pi)$ accepts. 
    \end{description}~\cite{Algorithms}
\end{defn}

\begin{defn}
    The set of all languages $L$ with the properties of definition~\ref{descr:np} is denoted by the complexity class NP~\cite{randomness}.
\end{defn}


\subsection{The class IP}
The class IP is an extension of the class NP. Two key differences are introduced: first, we allow randomized verification procedure by relaxing the requirements of definition~\ref{descr:np}. Then we allow interactive verification, i.e. we introduce a prover and a verifier.
\begin{defn}
    Consider the following model:
    \begin{itemize}
        \item a deterministic unbounded time prover $P$,
        \item a randomized polynomial time verifier $V$,
        \item a pair of conversation tapes on which $P$ and $V$ exchange information.
    \end{itemize}
    An interactive proof system (IPS) for a language $L$, is a protocol between $P$ and $V$ where
    \begin{itemize}
        \item $P$ and $V$ are given an input $x$,
        \item through an exchange of messages, $P$ tries to prove to $V$ that $x \in L$,
        \item at the end of the interaction, $V$ outputs either "accept" if the proof is satisfactory or "reject" if not.
    \end{itemize}
    We require that
    \begin{description}
\item[completeness:] if both $P$ and $V$ follow the protocol and $x \in L$, then
$$
\operatorname{Pr}[V \text { accepts } x] \geq c,
$$
\item[soundness:] if $x \notin L$ and $V$ follows the protocol, then, regardless of what $P$ does,
$$
\operatorname{Pr}[V \text { accepts } x] \leq s.
$$
    \end{description}
    Where $c$ is the completeness parameter and $s$ is the soundness parameter. A common setting is to take $c=\frac{2}{3}$ and $s=\frac{1}{3}$~\cite{randomness}.
\end{defn}

\begin{defn}
    The class \textbf{IP} is the class of languages L such that there exists an IPS for L~\cite{randomness}.
\end{defn}
In an interactive proof then, an all-powerful prover is trying to convince a skeptical, but computationally limited, verifier that a string $z$ (known to both) lies in the set $L$, even when it may be that in fact $z \notin L$. By interactively interrogating the prover, the verifier can reject false claims, i.e. determine with high statistical confidence whether $z \in L$ or not. Importantly, the verifier is allowed to probabilistically and adaptively choose its messages to the prover.


An example of an IP problem is the \textsc{graph non-isomorphism} problem.

We can prove that~\cite{ipequalspspace}:
\begin{theorem}
    \begin{equation}
 \text{IP} = \text{PSPACE}.
    \end{equation}
\end{theorem}

This means that we can polynomially verify the problems in PSPACE, using an interactive proof system as described above; those problems are (believed to be) vastly more difficult than those that can be checked using a static and deterministic proof (i.e. NP problems).


